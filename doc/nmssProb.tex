\documentclass[10pt,a4paper]{article}
\usepackage[utf8]{inputenc}
\usepackage{amsmath}
\usepackage{amsfonts}
\usepackage{amssymb}
\usepackage{graphicx}
\usepackage{mdwmath}
\usepackage{mdwtab}
\usepackage{xcolor}
\usepackage{url}
\usepackage{cite}
\usepackage{algorithmic}


\newcommand{\concat}{\mathbin{+\!\!+}}


\author{Yu Liu}
\title{\textsl{Not The Maximum Segment Sum} Problem \\ Introduction to Functional Programming}
 
\begin{document}
\maketitle

\textsl{Not the Maximum Segment Sum} (NMSS for short) problem is an interesting problem from the context of \cite{Bird}.
In this short report I introduce how to compute NMSS problem in \textbf{liner time} and also introduce some concepts and skills of functional programming (FP). As a related topic the \textsl{Maximum Segment Sum} problem has been studied by many researchers \cite{Cole93,HuIT96d,MMMH07}.

\section{Notations}
The syntax is mainly based on Haskell but a Java implementation is given.
List is denoted as  \([x,y,z]\) and $\concat$ means list concatenation. 
Function application is denoted with a space with its argument without parentheses, i.e., \(f~a\) equals to \(f~(a)\). 
Functions are curried and bound to left and thus \(f~a~b\) equals to \((f~a)~b\).
Functions can be composed by ``$\cdot$" and it has higher priority than other operators.
%\chapter{The Robotic Research and Brain Science}
\section{The Definition of Non-Segment}
By definition of segment, contiguous subsequence of a list,
the non-segment is defined: for a list at shortest with 3 elements, the non-segments are 
the sequences that are not contiguous subsequences (segments), formally, denoted by a regular expression
 as follows.
\[
F^*T^+F^+T(T+F)^*
\]
For example, given a list:
\[
[-4,-3,-7,+2,+1,-2,-1,-4]
\] 
The subsequence \([-4,-3,-7]\) is a segment, while \([-4,-7]\) is a not-segment and \([-4,-3,-7, +1]\) is also a not-segment of the original list.
We can use a automaton to recognize this regular expression:
\[
data State = E|S|M|N
\]
\begin{itemize}
	\item State E for \( F^*\) (empty)
	\item State S for \( F^*T^+\) (suffix)
	\item State M for \( F^*T^+F^+\) (middle)
	\item State N for \( F^*T^+F^+T(T+F)^*\)
\end{itemize}

The sum of a non-segment is called \textsl{Non-Segment Sum} (NSS).

\section{Specification}
The MNSS is to find a NSS which has the maximum sum.

\[
\begin{array}{lll}
mnss &::& [Int] \rightarrow Int \\
mnss &=& maximum \cdot map ~ sum \cdot nonsegs
\end{array}
\]

To define the \(nonsegs\) function, firstly, we introduce a $markings$ function:
\[
\begin{array}{lll}
markings  &::& [a] \rightarrow [[(a,Bool)]] \\
markings ~xs &=& [zip ~ xs ~ bs | bs \leftarrow booleans(length ~ xs)] \\
booleans ~ 0 &=& [[~]] \\
booleans ~ (n+1) &=&  [b:bs | b \leftarrow [True,False], bs \leftarrow booleans ~n]
\end{array}
\]

So that, the $nonsegs$ can be defined as 
\[
\begin{array}{lll}
nonsegs &::& [a] \rightarrow [[a]] \\
nonsegs &=& extarct \cdot filter ~ nonsegs \cdot markings \\
extarct &::& [[(a, Bool)]] \rightarrow [[a]] \\
extarct &=& map(map ~ fst \cdot filter ~ snd)
\end{array}
\] 

By making use of the automaton, $nonsegs$ can also be defined as
\[
nonseg  =  (\small{==} N) \cdot foldl~step~E \cdot map ~ snd
\]
Here, the \( Step ~ E\) means start the automaton form E state, and here are the states transformation functions:
\[
\begin{array}{lllll}
&step ~ E ~ False &= E \;  &step ~ M ~ False &= M  \\
&step ~ E ~ False &= S \;  &step ~ M ~ False &= N  \\
&step ~ S ~ False &= M \;  &step ~ N ~ False &= N  \\
&step ~ S ~ False &= S \;  &step ~ N ~ False &= N 
\end{array}
\]

\section{Derivation}


The \emph{mnss} can be redefined as 
\[
\begin{array}{lll}

mnss &=& maximum \cdot map ~ sum \cdot ~ extract \cdot filter ~ nonseg \cdot markings \\
extarct &=& map(map ~ fst \cdot filter ~ snd) \\
nonseg &=& (\small{==} N) \cdot foldl~step~E \cdot map ~ snd
\end{array}
\]
the problem is turned to find a way to apply the fusion law of $foldl$ to get a better algorithm,
that  \( extract \cdot filter ~ nonseg \cdot markings \) can be treated as an instance.
So, we can define a $pick$ function as follows.
\[
\begin{array}{lll}
pick :: State \rightarrow [a] \rightarrow [[a]] \\
pick ~ q = extract \cdot filter((==q) \cdot foldl ~ step ~ E \cdot map~ snd) \cdot markings
\end{array}
\]
just let q = N , \( nonsegs = pick~ N \).
Here clime these equations hold:
\[
\begin{array}{lll}
pick~ xs                  &=& [[~]] \\
pick~ S ~ [~]              &=& [~] \\
pick~ S ~(xs \concat [x] ) &=& map(\concat [x])(pick ~ S~xs \concat pick ~E~xs) \\
pick~ M ~[~] &=& [~] \\
pick~ M ~(xs \concat [x] ) &=& pick~M~xs \concat pick~S~xs \\
pick~ N ~[~] &=& [[~]] \\
pick~ N ~(xs \concat [x] ) &=& pick N xs \concat map(\concat [x])(pick ~ N~xs \concat pick ~M~xs) \\
\end{array}
\]

recast the definition of $pick$ as an instance of $foldl$:
firstly we make a tuple:
\[
pickall xs = (pick~E~xs, pick~S~xs,pick~M~xs,pick~N~xs)
\]
so, we have
\[
\begin{array}{lll}
pickall &=& foldl~ step([[~]],[~],[~],[~]) \\
step (ess,nss,mss,sss) x &=& (ess, map(\concat [x](sss \concat ess), \\
& \, &mss \concat sss, nss \concat map(\concat [x])(nss \concat mss) ) 
\end{array}
\]

so that \(mnss = maximum \cdot map~sum \cdot fourth \cdot pickall \). by define 
\[ 
tuple ~ f(w,x,y,z) = (f~w, f~x, f~y, f~z).
\]

Then we have :
\[
maximum \cdot map~ sum \cdot fourth = fourth \cdot tuple(maxmium \cdot map ~ sum)
\]

Then $mnss$ is changed to: \(mnss = fourth \cdot tuple( maximum \cdot map~sum)   \cdot pickall \)


By fusion law of $foldl$:

\[
f (foldl~ g~ a~xs) = foldl ~h~b~xs
\]
if \( f~a =b\) , and \(f(g~x~y)=h(f~x)y\) hold for all x and y.

f,g,and a are instantiations:
\[
\begin{array}{lll}
f &=& tuple(maximum \cdot map ~ sum) \\
g &=& step \\
a &=& ([[~]],[~],[~],[~])
\end{array}
\]

We need to find the h and b, satisfy the fusion conditions.
\[
tuple(maximum \cdot map ~ sum)([[~]],[~],[~],[~]) = (0,-\infty,-\infty,-\infty)
\]

Here h and b is given:
b is \((0,-\infty,-\infty,-\infty)\) and
\[
h(e,s,m,n) x = (e,(s \uparrow ~ e)+x ,m \uparrow s , n \uparrow( (n \uparrow m) +x) )
\]

so that 
\[
mnss = fourth \cdot foldl~h(0,-\infty,-\infty,-\infty)
\]

to simplify the definition of $mnss$ we can do this :
\[
\begin{array}{lll}
mnss ~ xs &=& fourth (foldl~h~(start(take~3~xs))~(drop~3~xs)) \\
start~[x,y,z] &=& (0, \uparrow[x+y+z,y+z,z], \uparrow[x,x+y,y], x+z)

\end{array}
\]

\section{Remarks}
Also, for at-least-length-k problem we can derive O(nk) algorithm.
And, for non-regular conditions such as \( F^*T^nF^*T^nF* ~ (n \geq 0)\) is susceptible
to the same method.
The Java implementation is here: \url{https://github.com/moyun/Mnss}.
Readers can evaluate the performance of the linear algorithm and compare with your own implementation.


\bibliographystyle{plain}
\bibliography{ref}	
	
\end{document}